\hypertarget{stm32l1xx__hal__i2c_8c}{\section{eps\-Subsystem/\-Drivers/\-S\-T\-M32\-L1xx\-\_\-\-H\-A\-L\-\_\-\-Driver/\-Src/stm32l1xx\-\_\-hal\-\_\-i2c.c File Reference}
\label{stm32l1xx__hal__i2c_8c}\index{eps\-Subsystem/\-Drivers/\-S\-T\-M32\-L1xx\-\_\-\-H\-A\-L\-\_\-\-Driver/\-Src/stm32l1xx\-\_\-hal\-\_\-i2c.\-c@{eps\-Subsystem/\-Drivers/\-S\-T\-M32\-L1xx\-\_\-\-H\-A\-L\-\_\-\-Driver/\-Src/stm32l1xx\-\_\-hal\-\_\-i2c.\-c}}
}


I2\-C H\-A\-L module driver. This file provides firmware functions to manage the following functionalities of the Inter Integrated Circuit (I2\-C) peripheral\-:  


{\ttfamily \#include \char`\"{}stm32l1xx\-\_\-hal.\-h\char`\"{}}\\*


\subsection{Detailed Description}
I2\-C H\-A\-L module driver. This file provides firmware functions to manage the following functionalities of the Inter Integrated Circuit (I2\-C) peripheral\-: \begin{DoxyAuthor}{Author}
M\-C\-D Application Team 
\end{DoxyAuthor}
\begin{DoxyVersion}{Version}
V1.\-1.\-3 
\end{DoxyVersion}
\begin{DoxyDate}{Date}
04-\/\-March-\/2016
\begin{DoxyItemize}
\item Initialization and de-\/initialization functions
\item I\-O operation functions
\item Peripheral State and Errors functions
\end{DoxyItemize}
\end{DoxyDate}
\begin{DoxyVerb}==============================================================================
                      ##### How to use this driver #####
==============================================================================
  [..]
  The I2C HAL driver can be used as follows:
  
  (#) Declare a I2C_HandleTypeDef handle structure, for example:
      I2C_HandleTypeDef  hi2c; 

  (#)Initialize the I2C low level resources by implementing the HAL_I2C_MspInit() API:
      (##) Enable the I2Cx interface clock
      (##) I2C pins configuration
          (+++) Enable the clock for the I2C GPIOs
          (+++) Configure I2C pins as alternate function open-drain
      (##) NVIC configuration if you need to use interrupt process
          (+++) Configure the I2Cx interrupt priority
          (+++) Enable the NVIC I2C IRQ Channel
      (##) DMA Configuration if you need to use DMA process
          (+++) Declare a DMA_HandleTypeDef handle structure for the transmit or receive channel
          (+++) Enable the DMAx interface clock using
          (+++) Configure the DMA handle parameters
          (+++) Configure the DMA Tx or Rx channel
          (+++) Associate the initialized DMA handle to the hi2c DMA Tx or Rx handle
          (+++) Configure the priority and enable the NVIC for the transfer complete interrupt on 
                the DMA Tx or Rx channel

  (#) Configure the Communication Speed, Duty cycle, Addressing mode, Own Address1,
      Dual Addressing mode, Own Address2, General call and Nostretch mode in the hi2c Init structure.

  (#) Initialize the I2C registers by calling the HAL_I2C_Init(), configures also the low level Hardware 
      (GPIO, CLOCK, NVIC...etc) by calling the customized HAL_I2C_MspInit(&hi2c) API.

  (#) To check if target device is ready for communication, use the function HAL_I2C_IsDeviceReady()

  (#) For I2C IO and IO MEM operations, three operation modes are available within this driver :

  *** Polling mode IO operation ***
  =================================
  [..]
    (+) Transmit in master mode an amount of data in blocking mode using HAL_I2C_Master_Transmit()
    (+) Receive in master mode an amount of data in blocking mode using HAL_I2C_Master_Receive()
    (+) Transmit in slave mode an amount of data in blocking mode using HAL_I2C_Slave_Transmit()
    (+) Receive in slave mode an amount of data in blocking mode using HAL_I2C_Slave_Receive()

  *** Polling mode IO MEM operation ***
  =====================================
  [..]
    (+) Write an amount of data in blocking mode to a specific memory address using HAL_I2C_Mem_Write()
    (+) Read an amount of data in blocking mode from a specific memory address using HAL_I2C_Mem_Read()


  *** Interrupt mode IO operation ***
  ===================================
  [..]
    (+) Transmit in master mode an amount of data in non-blocking mode using HAL_I2C_Master_Transmit_IT()
    (+) At transmission end of transfer, HAL_I2C_MasterTxCpltCallback() is executed and user can
         add his own code by customization of function pointer HAL_I2C_MasterTxCpltCallback()
    (+) Receive in master mode an amount of data in non-blocking mode using HAL_I2C_Master_Receive_IT()
    (+) At reception end of transfer, HAL_I2C_MasterRxCpltCallback() is executed and user can
         add his own code by customization of function pointer HAL_I2C_MasterRxCpltCallback()
    (+) Transmit in slave mode an amount of data in non-blocking mode using HAL_I2C_Slave_Transmit_IT()
    (+) At transmission end of transfer, HAL_I2C_SlaveTxCpltCallback() is executed and user can
         add his own code by customization of function pointer HAL_I2C_SlaveTxCpltCallback()
    (+) Receive in slave mode an amount of data in non-blocking mode using HAL_I2C_Slave_Receive_IT()
    (+) At reception end of transfer, HAL_I2C_SlaveRxCpltCallback() is executed and user can
         add his own code by customization of function pointer HAL_I2C_SlaveRxCpltCallback()
    (+) In case of transfer Error, HAL_I2C_ErrorCallback() function is executed and user can
         add his own code by customization of function pointer HAL_I2C_ErrorCallback()

  *** Interrupt mode IO MEM operation ***
  =======================================
  [..]
    (+) Write an amount of data in non-blocking mode with Interrupt to a specific memory address using
        HAL_I2C_Mem_Write_IT()
    (+) At Memory end of write transfer, HAL_I2C_MemTxCpltCallback() is executed and user can
         add his own code by customization of function pointer HAL_I2C_MemTxCpltCallback()
    (+) Read an amount of data in non-blocking mode with Interrupt from a specific memory address using
        HAL_I2C_Mem_Read_IT()
    (+) At Memory end of read transfer, HAL_I2C_MemRxCpltCallback() is executed and user can
         add his own code by customization of function pointer HAL_I2C_MemRxCpltCallback()
    (+) In case of transfer Error, HAL_I2C_ErrorCallback() function is executed and user can
         add his own code by customization of function pointer HAL_I2C_ErrorCallback()

  *** DMA mode IO operation ***
  ==============================
  [..]
    (+) Transmit in master mode an amount of data in non-blocking mode (DMA) using
        HAL_I2C_Master_Transmit_DMA()
    (+) At transmission end of transfer, HAL_I2C_MasterTxCpltCallback() is executed and user can
         add his own code by customization of function pointer HAL_I2C_MasterTxCpltCallback()
    (+) Receive in master mode an amount of data in non-blocking mode (DMA) using
        HAL_I2C_Master_Receive_DMA()
    (+) At reception end of transfer, HAL_I2C_MasterRxCpltCallback() is executed and user can
         add his own code by customization of function pointer HAL_I2C_MasterRxCpltCallback()
    (+) Transmit in slave mode an amount of data in non-blocking mode (DMA) using
        HAL_I2C_Slave_Transmit_DMA()
    (+) At transmission end of transfer, HAL_I2C_SlaveTxCpltCallback() is executed and user can
         add his own code by customization of function pointer HAL_I2C_SlaveTxCpltCallback()
    (+) Receive in slave mode an amount of data in non-blocking mode (DMA) using
        HAL_I2C_Slave_Receive_DMA()
    (+) At reception end of transfer, HAL_I2C_SlaveRxCpltCallback() is executed and user can
         add his own code by customization of function pointer HAL_I2C_SlaveRxCpltCallback()
    (+) In case of transfer Error, HAL_I2C_ErrorCallback() function is executed and user can
         add his own code by customization of function pointer HAL_I2C_ErrorCallback()

  *** DMA mode IO MEM operation ***
  =================================
  [..]
    (+) Write an amount of data in non-blocking mode with DMA to a specific memory address using
        HAL_I2C_Mem_Write_DMA()
    (+) At Memory end of write transfer, HAL_I2C_MemTxCpltCallback() is executed and user can
         add his own code by customization of function pointer HAL_I2C_MemTxCpltCallback()
    (+) Read an amount of data in non-blocking mode with DMA from a specific memory address using
        HAL_I2C_Mem_Read_DMA()
    (+) At Memory end of read transfer, HAL_I2C_MemRxCpltCallback() is executed and user can
         add his own code by customization of function pointer HAL_I2C_MemRxCpltCallback()
    (+) In case of transfer Error, HAL_I2C_ErrorCallback() function is executed and user can
         add his own code by customization of function pointer HAL_I2C_ErrorCallback()


   *** I2C HAL driver macros list ***
   ==================================
   [..]
     Below the list of most used macros in I2C HAL driver.

    (+) __HAL_I2C_ENABLE:      Enable the I2C peripheral
    (+) __HAL_I2C_DISABLE:     Disable the I2C peripheral
    (+) __HAL_I2C_GET_FLAG:    Check whether the specified I2C flag is set or not
    (+) __HAL_I2C_CLEAR_FLAG:  Clear the specified I2C pending flag
    (+) __HAL_I2C_ENABLE_IT:   Enable the specified I2C interrupt
    (+) __HAL_I2C_DISABLE_IT:  Disable the specified I2C interrupt

   [..]
     (@) You can refer to the I2C HAL driver header file for more useful macros\end{DoxyVerb}


\begin{DoxyAttention}{Attention}

\end{DoxyAttention}
\subsubsection*{\begin{center}\copyright{} C\-O\-P\-Y\-R\-I\-G\-H\-T(c) 2016 S\-T\-Microelectronics\end{center} }

Redistribution and use in source and binary forms, with or without modification, are permitted provided that the following conditions are met\-:
\begin{DoxyEnumerate}
\item Redistributions of source code must retain the above copyright notice, this list of conditions and the following disclaimer.
\item Redistributions in binary form must reproduce the above copyright notice, this list of conditions and the following disclaimer in the documentation and/or other materials provided with the distribution.
\item Neither the name of S\-T\-Microelectronics nor the names of its contributors may be used to endorse or promote products derived from this software without specific prior written permission.
\end{DoxyEnumerate}

T\-H\-I\-S S\-O\-F\-T\-W\-A\-R\-E I\-S P\-R\-O\-V\-I\-D\-E\-D B\-Y T\-H\-E C\-O\-P\-Y\-R\-I\-G\-H\-T H\-O\-L\-D\-E\-R\-S A\-N\-D C\-O\-N\-T\-R\-I\-B\-U\-T\-O\-R\-S \char`\"{}\-A\-S I\-S\char`\"{} A\-N\-D A\-N\-Y E\-X\-P\-R\-E\-S\-S O\-R I\-M\-P\-L\-I\-E\-D W\-A\-R\-R\-A\-N\-T\-I\-E\-S, I\-N\-C\-L\-U\-D\-I\-N\-G, B\-U\-T N\-O\-T L\-I\-M\-I\-T\-E\-D T\-O, T\-H\-E I\-M\-P\-L\-I\-E\-D W\-A\-R\-R\-A\-N\-T\-I\-E\-S O\-F M\-E\-R\-C\-H\-A\-N\-T\-A\-B\-I\-L\-I\-T\-Y A\-N\-D F\-I\-T\-N\-E\-S\-S F\-O\-R A P\-A\-R\-T\-I\-C\-U\-L\-A\-R P\-U\-R\-P\-O\-S\-E A\-R\-E D\-I\-S\-C\-L\-A\-I\-M\-E\-D. I\-N N\-O E\-V\-E\-N\-T S\-H\-A\-L\-L T\-H\-E C\-O\-P\-Y\-R\-I\-G\-H\-T H\-O\-L\-D\-E\-R O\-R C\-O\-N\-T\-R\-I\-B\-U\-T\-O\-R\-S B\-E L\-I\-A\-B\-L\-E F\-O\-R A\-N\-Y D\-I\-R\-E\-C\-T, I\-N\-D\-I\-R\-E\-C\-T, I\-N\-C\-I\-D\-E\-N\-T\-A\-L, S\-P\-E\-C\-I\-A\-L, E\-X\-E\-M\-P\-L\-A\-R\-Y, O\-R C\-O\-N\-S\-E\-Q\-U\-E\-N\-T\-I\-A\-L D\-A\-M\-A\-G\-E\-S (I\-N\-C\-L\-U\-D\-I\-N\-G, B\-U\-T N\-O\-T L\-I\-M\-I\-T\-E\-D T\-O, P\-R\-O\-C\-U\-R\-E\-M\-E\-N\-T O\-F S\-U\-B\-S\-T\-I\-T\-U\-T\-E G\-O\-O\-D\-S O\-R S\-E\-R\-V\-I\-C\-E\-S; L\-O\-S\-S O\-F U\-S\-E, D\-A\-T\-A, O\-R P\-R\-O\-F\-I\-T\-S; O\-R B\-U\-S\-I\-N\-E\-S\-S I\-N\-T\-E\-R\-R\-U\-P\-T\-I\-O\-N) H\-O\-W\-E\-V\-E\-R C\-A\-U\-S\-E\-D A\-N\-D O\-N A\-N\-Y T\-H\-E\-O\-R\-Y O\-F L\-I\-A\-B\-I\-L\-I\-T\-Y, W\-H\-E\-T\-H\-E\-R I\-N C\-O\-N\-T\-R\-A\-C\-T, S\-T\-R\-I\-C\-T L\-I\-A\-B\-I\-L\-I\-T\-Y, O\-R T\-O\-R\-T (I\-N\-C\-L\-U\-D\-I\-N\-G N\-E\-G\-L\-I\-G\-E\-N\-C\-E O\-R O\-T\-H\-E\-R\-W\-I\-S\-E) A\-R\-I\-S\-I\-N\-G I\-N A\-N\-Y W\-A\-Y O\-U\-T O\-F T\-H\-E U\-S\-E O\-F T\-H\-I\-S S\-O\-F\-T\-W\-A\-R\-E, E\-V\-E\-N I\-F A\-D\-V\-I\-S\-E\-D O\-F T\-H\-E P\-O\-S\-S\-I\-B\-I\-L\-I\-T\-Y O\-F S\-U\-C\-H D\-A\-M\-A\-G\-E. 